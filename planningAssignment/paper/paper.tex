\def\year{2015}
%File: formatting-instruction.tex
\documentclass[letterpaper]{article}
\usepackage{aaai}
\usepackage{times}
\usepackage{helvet}
\usepackage{courier}
\usepackage[utf8]{inputenc}
\frenchspacing
\setlength{\pdfpagewidth}{8.5in}
\setlength{\pdfpageheight}{11in}
\pdfinfo{
/Title (Planejamento em um Mundo Baseado em Portal (Valve))
/Author (Guilherme de M. M. Taschetto)}
\setcounter{secnumdepth}{0}
\begin{document}

\title{Planejamento em um Mundo\\Baseado em Portal (Valve)}
\author{Guilherme de M. M. Taschetto\\
Acadêmico da Faculdade de Informática da\\
Pontifícia Universidade Católica do Rio Grande do Sul\\
Av. Ipiranga, 6681\\
Porto Alegre, Rio Grande do Sul 90619-900\\
}

\maketitle
\begin{abstract}
Este artigo apresenta a modelagem, em linguagem PDDL, de um mundo (domínio) baseado nas regras do jogo Portal (desenvolvido pela Valve) e de fases (problemas) dentro do contexto do domínio. Após, apresenta-se uma análise dos resultados obtidos ao solucionar os problemas com o algoritmo Graph Plan, implementado pela ferramenta JavaGP.
\end{abstract}

\section{Introdução}

Este artigo apresenta uma modelagem, em linguagem PDDL, de um mundo (domínio) baseado nas regras do jogo Portal (desenvolvido pela Valve). Este mundo é composto por localizações, podendo elas serem salas ou corredores. Além disso, um ou mais cubos podem estar posicionados nestas salas e corredores.

Neste contexto, o robô ATLAS deve realizar uma tarefa: à partir da sua posição inicial, recolher cada cubo e levá-lo até a uma posição de destino, i. e., uma sala específica. Porém, Atlas deve seguir algumas regras. São elas:

\begin{itemize}
\item ATLAS pode mover-se livremente entre corredores adjacentes, i. e., que possuem conexão entre si;
\item Caso esteja em um corredor, ATLAS pode entrar em qualquer sala que possua conexão com o corredor em que se encontra;
\item Caso esteja em uma sala, ATLAS pode sair da sala para qualquer corredor que possua conexão com a sala em que se encontra;
\item ATLAS não pode mover-se de uma sala para a outra diretamente, i. e., para ir de uma sala à outra ATLAS deve obrigatoriamente passar por um corredor intermediário;
\item Caso esteja na mesma sala ou corredor que algum cubo e não esteja carregando algum cubo, ATLAS pode pegar o cubo que está na sala e levar consigo;
\item Caso esteja de posse de um cubo, ATLAS pode largá-lo na sua localização atual.
\end{itemize}

\section{Formalização do Domínio}

À partir da especificação descritiva do domínio são definidos três subconjuntos descritivos:

\begin{itemize}
\item Tipos
\item Predicados
\item Ações
\end{itemize}

\subsection{Tipos}

No contexto do domínio em questão, a seguinte hierarquia de tipos polimórficos foi definida:

\begin{description}
\item[\texttt{robot location cube - object}] \hfill\\ \texttt{robot}, \texttt{location} e \texttt{cube} são especializações do tipo básico \texttt{object}.
\item[\texttt{hallway room - location}] \hfill\\ \texttt{hallway} e \texttt{room} especializam o tipo \texttt{location}.
\end{description}

O polimorfismo será útil na hora de utilizar parâmetros tipados nos predicados e ações.

\subsection{Predicados}

Com base na especificação de domínio e nos tipos previamente definidos, alguns predicados são necessários compor um estado válido do domínio. São eles:

\begin{description}
\item[\texttt{at ?r - robot ?loc - location}] \hfill\\ Verdadeiro caso o robô esteja na localização especificada.
\item[\texttt{connected ?loc1 ?loc2 - location}] \hfill\\ Verdadeiro caso as duas localizações especificadas possuam conexão entre si.
\item[\texttt{in ?c - cube ?loc - location}] \hfill\\ Verdadeiro caso um cubo esteja na localização especificada.
\item[\texttt{has ?r - robot ?c - cube}] \hfill\\ Verdadeiro caso o robô esteja carregando o cubo específicado.
\item[\texttt{unloaded ?r - robot}] \hfill\\ Verdadeiro caso o robô não esteja carregando nenhum cubo.
\end{description}

\subsection{Ações}

\begin{verbatim}
(:action enter
  :parameters (
    ?r    - robot
    ?from - hallway
    ?to   - room)
  :precondition
    (and (at ?r ?from)
         (connected ?from ?to))
  :effect
    (and (not (at ?r ?from))
         (at ?r ?to)))
\end{verbatim}

\begin{verbatim}
(:action exit
  :parameters (
    ?r    - robot
    ?from - room
    ?to   - hallway)
  :precondition
    (and (at ?r ?from)
         (connected ?from ?to))
  :effect
    (and (not (at ?r ?from))
         (at ?r ?to)))
\end{verbatim}

\begin{verbatim}
(:action move
  :parameters (
    ?r        - robot
    ?from ?to - hallway)
  :precondition
    (and (at ?r ?from)
         (connected ?from ?to))
  :effect
    (and (not (at ?r ?from))
         (at ?r ?to)))
\end{verbatim}

\begin{verbatim}
(:action pickup
  :parameters (
    ?r   - robot
    ?c   - cube
    ?loc - location)
  :precondition
    (and (at ?r ?loc)
         (in ?c ?loc)
         (not (has ?r ?c))
         (unloaded ?r))
  :effect
    (and (has ?r ?c)
         (not (in ?c ?loc))
         (not (unloaded ?r))))
\end{verbatim}

\begin{verbatim}
(:action drop
  :parameters (
    ?r   - robot
    ?c   - cube
    ?loc - location)
  :precondition
    (and (at ?r ?loc)
         (has ?r ?c)
         (not (unloaded ?r)))
  :effect
    (and (not (has ?r ?c))
         (in ?c ?loc)
         (unloaded ?r)))
\end{verbatim}

\section{Formalização dos Problemas}

Problemas.

\section{Experimentação}

Experimentação.

PB1 6   317ms
PB2 24  4902ms
PB3 23  7801ms
PB4 6   369ms
PB5 14  1082ms
PB6 24  22738ms

\section{Conclusão}

Conclusão.

\end{document}
